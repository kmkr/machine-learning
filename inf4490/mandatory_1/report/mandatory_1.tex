\documentclass{article}

\usepackage{booktabs}
\usepackage{microtype}
\title{Mandatory Assignment 1}
\author{Kris-Mikael Krister (krismikk)\\\texttt{krismikael@protonmail.com}}
\date{\today}

\begin{document}

\maketitle
\section*{Exhaustive Search}

The shortest path is Copenhagen $\rightarrow$ Hamburg $\rightarrow$ Brussels $\rightarrow$ Dublin $\rightarrow$ Barcelona $\rightarrow$ Belgrade $\rightarrow$ Istanbul $\rightarrow$ Bucharest $\rightarrow$ Budapest $\rightarrow$ Berlin. The total distance for the route is 7486.31 km.\\

\noindent The solution was found using 21.25 seconds on a Linux i5-4670 CPU @ 3.40GHz. A single core was used when running the program, so exploiting the multi-core architecture would be a possible improvement to the program.\\

\begin{center}
\begin{tabular}{crr}
\toprule
Number of cities & Permutations & Execution time (seconds) \\
\midrule
$5$ & $120$ & $0.000349998474121$ \\
$6$ & $720$ & $0.00444889068604$ \\
$7$ & $5\,040$ & $0.0201399326324$ \\
$8$ & $40\,320$ & $0.219367980957$ \\
$9$ & $362\,880$ & $1.87963008881$ \\
$10$ & $3\,628\,800$ & $21.25$ \\
$11$ & $39\,916\,800$ & $258$ \\
$12$ & $479\,001\,600$ & $3406.87$ \\
\bottomrule
\end{tabular}
\end{center}

\noindent Difference in execution time corresponds to the number of permutations. The amount of permutations is factorial ($O(n!)$), so based on the results above, the expected running time for all 24 cities would be approximately $3406.87 * 13 * 14 * 15 * 16 * 17 * 18 * 19 * 20 * 21 * 22 * 23 * 24 \approx 4.41 \times 10^{18}$ seconds.

\subsection*{How to run the program}

Execute the \texttt{tsp\_exhaustive.py} file with the amount of cities to check.

\begin{verbatim}
$ python tsp_exhaustive.py 5

Exhastive search using 5 first cities
Found shortest {'distance': 4983.38, 'route': ('Barcelona',
'Belgrade', 'Bucharest', 'Berlin', 'Brussels')} in
time 0.00048303604126
\end{verbatim}

\noindent All modules have one or more unit tests to verify the implementation. To run all tests, use Python's unittest discovery:

\begin{verbatim}
$ python -m unittest discover -p '*_test.py'
\end{verbatim}

\end{document}
